\documentclass[draftclsnofoot,onecolumn]{IEEEtran}

\usepackage{graphicx}                                        
\usepackage{amssymb}                                         
\usepackage{amsmath}                                         
\usepackage{amsthm}         


\usepackage{alltt}                                           
\usepackage{float}
\usepackage{color}
\usepackage{url}

\usepackage{balance}
\usepackage[TABBOTCAP, tight]{subfigure}
\usepackage{enumitem}
\usepackage{pstricks, pst-node}

\usepackage{geometry}


\geometry{textheight=8.5in, textwidth=6in}

%random comment

\newcommand{\cred}[1]{{\color{red}#1}}
\newcommand{\cblue}[1]{{\color{blue}#1}}

\usepackage{hyperref}
\usepackage{geometry}

\usepackage{longtable}


\newcommand{\longtableendfoot}{Please continue at the next page}

\def\name{Andrew Chase}

\date{}
\author{\name \\ 
OS II \\
Spring 2015 \\
Abstract: Describes work done to build and run a virtual kernel, as well as develop and test a simple concurrency demo.
}
\title{Project 1: Getting Acquainted}

%pull in the necessary preamble matter for pygments output
\input{pygments.tex}

%% The following metadata will show up in the PDF properties
\hypersetup{
  colorlinks = true,
  urlcolor = black,
  pdfauthor = {\name},
  pdfkeywords = {cs311 ``operating systems'' files filesystem I/O},
  pdftitle = {CS 311 Project 1: UNIX File I/O},
  pdfsubject = {CS 311 Project 1},
  pdfpagemode = UseNone
}

\begin{document}

\maketitle

\pagebreak

\tableofcontents

\pagebreak

\section{Work Log}

\begin{tabular}{ll}
  \textbf{Date} & \textbf{Work Done} \\
  Apr 7 & Build Kernel and run on os-class \\
  Apr 8 & Write Concurrency solution \\
  April 11& Write this paper
\end{tabular}

\section{Kernel Setup}

\subsection{Command List}

The following was the bash history for setting up the Kernel.

\begin{Verbatim}[commandchars=\\\{\},numbers=left,firstnumber=1,stepnumber=1]
\PY{n+nb}{cd} /scratch/spring2016/
ls
mkdir cs444\PYZhy{}017
ls
\PY{n+nb}{cd }cs444\PYZhy{}017
ls
git init
git clone git://git.yoctoproject.org/linux\PYZhy{}yocto\PYZhy{}3.14
git clone \PYZhy{}\PYZhy{}depth \PY{l+m}{1} git://git.yoctoproject.org/linux\PYZhy{}yocto\PYZhy{}3.14 v3.14.26
ls
\PY{n+nb}{cd }v3.14.26/
ls
\PY{n+nb}{source} /scratch/opt/environment\PYZhy{}setup\PYZhy{}i586\PYZhy{}poky\PYZhy{}linux
less /scratch/opt/environment\PYZhy{}setup\PYZhy{}i586\PYZhy{}poky\PYZhy{}linux
make
make \PYZhy{}j4 all
cp \PYZhy{}r /scratch/spring2015/files/config\PYZhy{}3.14.26\PYZhy{}yocto\PYZhy{}qemu .config
\PY{n+nb}{cd} .config
ls
man cp
cp \PYZhy{}r /scratch/spring2015/files/config\PYZhy{}3.14.26\PYZhy{}yocto\PYZhy{}qemu .config/
cp \PYZhy{}r /scratch/spring2015/files/config\PYZhy{}3.14.26\PYZhy{}yocto\PYZhy{}qemu .config
less .config
make \PYZhy{}j4 all
make clean
make \PYZhy{}j4 all
\PY{n+nb}{exit}
ssh chasean@os\PYZhy{}class.engr.oregonstate.edu
\PY{n+nb}{exit}
ls
\PY{n+nb}{echo} \PY{l+s+s2}{\PYZdq{}cd /scratch/spring2016/cs444\PYZhy{}017\PYZdq{}} \PYZgt{} os
chmod a+x os
./os
ls
. os
ls
\PY{n+nb}{cd}
ls
ls \PYZhy{}a
vi .profile
vi .bash\PYZus{}profile 
vi .bashrc
\PY{n+nb}{exit}
vi .bashr
vi .bashrc
\PY{n+nb}{exit}
ssh chasean@os\PYZhy{}class.engr.oregonstate.edu
. os
ls
vi start\PYZhy{}qemu
\PY{c}{\PYZsh{} qemu\PYZhy{}system\PYZhy{}i386 \PYZhy{}gdb tcp::5517 \PYZhy{}S \PYZhy{}nographic \PYZhy{}kernel arch/x86/boot/bzImage \PYZhy{}drive file=core\PYZhy{}image\PYZhy{}lsb\PYZhy{}sdk\PYZhy{}qemux86.ext3,if=virtio \PYZhy{}enable\PYZhy{}kvm \PYZhy{}net none \PYZhy{}usb \PYZhy{}localtime \PYZhy{}\PYZhy{}no\PYZhy{}reboot \PYZhy{}\PYZhy{}append \PYZdq{}root=/dev/vda rw console=ttyS0 debug\PYZdq{}}
cp /scratch/spring2016/files/bzImage\PYZhy{}qemux86.bin .
cp /scratch/spring2016/files/core\PYZhy{}image\PYZhy{}lsb\PYZhy{}sdk\PYZhy{}qemux86.ext3 .
ls
\PY{n+nb}{cd }v3.14.26/
ls
\PY{n+nb}{cd }kernel/
ls
\PY{n+nb}{cd} ..
ls
less README 
\PY{n+nb}{cd }kernel/
ls
\PY{n+nb}{cd} ..
ls
less Kbuild 
less Makefile 
\PY{n+nb}{cd} ..
ls
tree 
tree  \PYZgt{} t
less t
\PY{n+nb}{cd }v3.14.26/
ls
\PY{n+nb}{cd} ..
ls
less t
rm t
mv start\PYZhy{}qemu core\PYZhy{}image\PYZhy{}lsb\PYZhy{}sdk\PYZhy{}qemux86.ext3 bzImage\PYZhy{}qemux86.bin v3.14.26/
ls
\PY{n+nb}{cd }v3.14.26/
. start\PYZhy{}qemu 
\PY{n+nb}{source} /scratch/opt/environment\PYZhy{}setup\PYZhy{}i586\PYZhy{}poky\PYZhy{}linux
. start\PYZhy{}qemu
ls
. os
\PY{n+nb}{cd }v3.14.26/
. /scratch/opt/environment\PYZhy{}setup\PYZhy{}i586\PYZhy{}poky\PYZhy{}linux
\PY{n+nv}{\PYZdl{}GDB} 5517
man gdb
ls
\PY{n+nb}{echo} \PY{n+nv}{\PYZdl{}GDB}
vi \PY{n+nv}{\PYZdl{}GDB}
vi \PY{l+s+sb}{`}\PY{n+nb}{echo} \PY{n+nv}{\PYZdl{}GDB}\PY{l+s+sb}{`}
which \PY{l+s+sb}{`}\PY{n+nb}{echo} \PY{n+nv}{\PYZdl{}GDB}\PY{l+s+sb}{`}
/scratch/opt/sysroots/x86\PYZus{}64\PYZhy{}pokysdk\PYZhy{}linux/usr/bin/i586\PYZhy{}poky\PYZhy{}linux/i586\PYZhy{}poky\PYZhy{}linux\PYZhy{}gdb \PYZhy{}\PYZhy{}help
gdb 127.0.0.1:5517 vmlinux
ls
gdb 127.0.0.1:5517 bzImage\PYZhy{}qemux86.bin 
ls
gdb 127.0.0.1:5517 
gdb 127.0.0.1:5517 core\PYZhy{}image\PYZhy{}lsb\PYZhy{}sdk\PYZhy{}qemux86.ext3 
ls
\PY{n+nv}{\PYZdl{}GDB} 127.0.0.1:5517
\PY{n+nv}{\PYZdl{}GDB} 127.0.0.1:5517 bzImage\PYZhy{}qemux86.bin 
vi start\PYZhy{}qemu 
screen \PYZhy{}R q
less
ps \PYZhy{}u chasan
ps 
ps \PYZhy{}u chasean
\PY{n+nv}{\PYZdl{}GDB} 127.0.0.1:5517 2018
. /scratch/opt/environment\PYZhy{}setup\PYZhy{}i586\PYZhy{}poky\PYZhy{}linux
\PY{n+nv}{\PYZdl{}GDB} 127.0.0.1:5517 2018
\PY{n+nv}{\PYZdl{}GDB} :5517 2018
\PY{n+nv}{\PYZdl{}GDB} 2018
\PY{n+nv}{\PYZdl{}GDB} 127.0.0.1:5517 
\PY{n+nv}{\PYZdl{}GDB} \PYZhy{}\PYZhy{}help \PY{p}{|} less
ls
. os
ls
\PY{n+nb}{cd }v3.14.26/
\PY{n+nv}{\PYZdl{}GDB} 127.0.0.1:5517 arch/x86/boot/bzImage 
\PY{n+nv}{\PYZdl{}GDB}
ps
ps \PYZhy{}u chasean
\PY{n+nb}{kill }2018
\PY{n+nb}{exit}
. start\PYZhy{}qemu 
. /scratch/opt/environment\PYZhy{}setup\PYZhy{}i586\PYZhy{}poky\PYZhy{}linux
. start\PYZhy{}qemu 
vi start\PYZhy{}qemu 
. start\PYZhy{}qemu 
\PY{n+nb}{exit}
screen \PYZhy{}r q
\PY{n+nb}{exit}
\end{Verbatim}


\subsection{Qemu command line flags}

\begin{tabular}{ll}
  \textbf{Flag} & \textbf{Meaning} \\
  -gdb tcp::???? & Wait for gdb connection on tcp:???? before continuing \\
  -S & Freeze CPU execution on startup (use gdb to continue) \\
  -nographic & QEMU has a graphical output system. This flag disables it \\
  -kernel bzImage-qemux86.bin & Use selected bzimage as the system Kernel \\
  -drive ... & Use selected drive and input file as the virtual hard drive for the machine  \\
  -enable-kvm & Enabled hardware assisted virtualization using special kernel-level optimizations \\
  -net none & Disable networking \\
  -usb & Enable usb driver support \\
  -localtime & Sets virtual clock to the system clock; Deprecated 2009 in qemu commit 1ed2fc1 \\
  -no-reboot & When system shuts down, exit qemu instead of rebooting the virtual machine \\
  -append ... & Kernels accept command line options, use these options specifed
\end{tabular}

\section{Concurrency Exercise: Questions}

\subsection{What do you think the main point of this assignment is?}

I think the main point of the assignment is to review writing, compiling, and testing low-level code which students will be doing in this course.

\subsection{How did you personally approach the problem? Design decisions, algorithm, etc.}

I used the low-level Pipe construct, after making sure that it was thread-safe for the purposes of the assignment. The use of pipes is simple and easy for programmers to understand. The trade-off for pipes was that it only works in a thread-safe way for small message sizes and is tied to a specific operating system level api, but the advantage of a pipe are that they are very easy to understand and implement.

\subsection{How did you ensure your solution was correct? Testing details, for instance.}

I ran the solution and inserted system print log statements to make sure the system was behaving as desired.

\subsection{What did you learn?}

That pipes can be used for concurrency, that newer Intel chips have randomization instructions, and a lot about the cmake build tool.

\section{Concurrency Exercise: File Tree}

\begin{verbatim}
.
| - CMakeLists.txt
| - get_random.c
| - get_random.h
| - main.c
| - vendor
    | - drng
    |�� | - LICENSE
    |�� | - README
    |�� | - config.h
    |�� | - cpuid.c
    |�� | - cpuid.h
    |�� | - drng.c
    |�� | - drng.h
    | - mt19937ar.c
    | - mt19937ar.h
    | - not_using_drng
    |�� | - using_drng.h
    | - test_drng
    |�� | - test.c
    | - test_drng_bin
    |�� | - CMakeFiles
    |��     | - CMakeTmp
    | - using_drng
        | - using_drng.h
        

8 directories, 16 files
\end{verbatim}

\section{Concurrency Exercise: Git Log}

%% This file was generated by the script latex-git-log
\begin{tabular}{lp{12cm}}
  \label{tabular:legend:git-log}
  \textbf{acronym} & \textbf{meaning} \\
  V & \texttt{version} \\
  tag & \texttt{git tag} \\
  MF & Number of \texttt{modified files}. \\
  AL & Number of \texttt{added lines}. \\
  DL & Number of \texttt{deleted lines}. \\
\end{tabular}

\bigskip

\iflanguage{ngerman}{\shorthandoff{"}}{}
\begin{longtable}{|rlllrrr|}
\hline \multicolumn{1}{|c}{\textbf{V}} & \multicolumn{1}{c}{\textbf{tag}}
& \multicolumn{1}{c}{\textbf{date}}
& \multicolumn{1}{c}{\textbf{commit message}} & \multicolumn{1}{c}{\textbf{MF}}
& \multicolumn{1}{c}{\textbf{AL}} & \multicolumn{1}{c|}{\textbf{DL}} \\ \hline
\endhead

\hline \multicolumn{7}{|r|}{\longtableendfoot} \\ \hline
\endfoot

\hline% \hline
\endlastfoot

\hline 1 &  & 2016-04-07 & init & 2 & 15 & 0 \\
\hline 2 &  & 2016-04-07 & Add pthreads from pthreads example & 3 & 42 & 11 \\
\hline 3 &  & 2016-04-07 & Stub consumers/producers functions and add pipe. & 1 & 47 & 14 \\
\hline 4 &  & 2016-04-07 & Add random vendor files and interface & 12 & 1112 & 2 \\
\hline 5 &  & 2016-04-07 & Implement randomness into producer & 1 & 18 & 5 \\
\hline 6 &  & 2016-04-08 & Use Try\_Compile to decide random at compile time & 6 & 50 & 14 \\
\end{longtable}


\section{Concurrency Exercise: Code Listings}

\subsection{main.c}

\begin{Verbatim}[commandchars=\\\{\}]
\PY{c+cm}{/*}


\PY{c+cm}{REFERENCES:}
\PY{c+cm}{* https://computing.llnl.gov/tutorials/pthreads/samples/hello.c}
\PY{c+cm}{* http://mij.oltrelinux.com/devel/unixprg/}
\PY{c+cm}{* https://stackoverflow.com/questions/1620918/cmake\PYZhy{}and\PYZhy{}libpthread}
\PY{c+cm}{* http://www.tutorialspoint.com/cprogramming/c\PYZus{}structures.htm}
\PY{c+cm}{* https://stackoverflow.com/questions/12657962/how\PYZhy{}do\PYZhy{}i\PYZhy{}generate\PYZhy{}a\PYZhy{}random\PYZhy{}number\PYZhy{}between\PYZhy{}two\PYZhy{}variables\PYZhy{}that\PYZhy{}i\PYZhy{}have\PYZhy{}stored}
\PY{c+cm}{* https://stackoverflow.com/questions/4975340/int\PYZhy{}to\PYZhy{}unsigned\PYZhy{}int\PYZhy{}conversion}

\PY{c+cm}{*/}
\PY{c+cp}{\PYZsh{}}\PY{c+cp}{include \PYZlt{}pthread.h\PYZgt{}}
\PY{c+cp}{\PYZsh{}}\PY{c+cp}{include \PYZlt{}stdio.h\PYZgt{}}
\PY{c+cp}{\PYZsh{}}\PY{c+cp}{include \PYZlt{}stdlib.h\PYZgt{}}
\PY{c+cp}{\PYZsh{}}\PY{c+cp}{include \PYZlt{}unistd.h\PYZgt{}}
\PY{c+cp}{\PYZsh{}}\PY{c+cp}{include \PYZdq{}get\PYZus{}random.h\PYZdq{}}

\PY{c+cp}{\PYZsh{}}\PY{c+cp}{define NUM\PYZus{}THREADS    2}

\PY{k}{struct} \PY{n}{WorkUnit} \PY{p}{\PYZob{}}
    \PY{k+kt}{unsigned} \PY{k+kt}{int} \PY{n}{unitNumber}\PY{p}{;}
    \PY{k+kt}{unsigned} \PY{k+kt}{int} \PY{n}{workTime}\PY{p}{;}
\PY{p}{\PYZcb{}}\PY{p}{;}

\PY{k+kt}{int} \PY{n}{workPipe}\PY{p}{[}\PY{l+m+mi}{2}\PY{p}{]}\PY{p}{;}

\PY{c+cp}{\PYZsh{}}\PY{c+cp}{define readPipe workPipe[0]}
\PY{c+cp}{\PYZsh{}}\PY{c+cp}{define writePipe workPipe[1]}

\PY{k+kt}{void} \PY{o}{*}\PY{n+nf}{consumer}\PY{p}{(}\PY{k+kt}{void} \PY{o}{*}\PY{n}{threadId}\PY{p}{)} \PY{p}{\PYZob{}}
    \PY{k}{struct} \PY{n}{WorkUnit} \PY{n}{work\PYZus{}message}\PY{p}{;}
    \PY{k+kt}{long} \PY{n}{tid}\PY{p}{;}
    \PY{k+kt}{ssize\PYZus{}t} \PY{n}{ret}\PY{p}{;}
    \PY{n}{tid} \PY{o}{=} \PY{p}{(}\PY{k+kt}{long}\PY{p}{)} \PY{n}{threadId}\PY{p}{;}

    \PY{k}{for} \PY{p}{(}\PY{p}{;} \PY{p}{;}\PY{p}{)} \PY{p}{\PYZob{}}
        \PY{n}{ret} \PY{o}{=} \PY{n}{read}\PY{p}{(}\PY{n}{readPipe}\PY{p}{,} \PY{o}{\PYZam{}}\PY{n}{work\PYZus{}message}\PY{p}{,} \PY{k}{sizeof}\PY{p}{(}\PY{k}{struct} \PY{n}{WorkUnit}\PY{p}{)}\PY{p}{)}\PY{p}{;}
        \PY{k}{if} \PY{p}{(}\PY{n}{ret} \PY{o}{=}\PY{o}{=} \PY{l+m+mi}{0}\PY{p}{)}
            \PY{k}{break}\PY{p}{;}
        \PY{n}{sleep}\PY{p}{(}\PY{n}{work\PYZus{}message}\PY{p}{.}\PY{n}{workTime}\PY{p}{)}\PY{p}{;}
        \PY{n}{printf}\PY{p}{(}\PY{l+s}{\PYZdq{}}\PY{l+s}{Thread \PYZsh{}\PYZpc{}ld consumed \PYZpc{}u. Took \PYZpc{}u seconds.}\PY{l+s+se}{\PYZbs{}n}\PY{l+s}{\PYZdq{}}\PY{p}{,} \PY{n}{tid}\PY{p}{,} \PY{n}{work\PYZus{}message}\PY{p}{.}\PY{n}{unitNumber}\PY{p}{,} \PY{n}{work\PYZus{}message}\PY{p}{.}\PY{n}{workTime}\PY{p}{)}\PY{p}{;}

    \PY{p}{\PYZcb{}}\PY{p}{;}
    \PY{n}{pthread\PYZus{}exit}\PY{p}{(}\PY{n+nb}{NULL}\PY{p}{)}\PY{p}{;}
\PY{p}{\PYZcb{}}

\PY{c+cp}{\PYZsh{}}\PY{c+cp}{pragma clang diagnostic push}
\PY{c+cp}{\PYZsh{}}\PY{c+cp}{pragma clang diagnostic ignored \PYZdq{}\PYZhy{}Wmissing\PYZhy{}noreturn\PYZdq{}}

\PY{k+kt}{void} \PY{n+nf}{producer}\PY{p}{(}\PY{p}{)} \PY{p}{\PYZob{}}
    \PY{k}{struct} \PY{n}{WorkUnit} \PY{n}{producerMessage}\PY{p}{;}
    \PY{k+kt}{unsigned} \PY{k+kt}{int} \PY{n}{producerWaitTime}\PY{p}{;}
    \PY{k}{for} \PY{p}{(}\PY{p}{;} \PY{p}{;}\PY{p}{)} \PY{p}{\PYZob{}}
        \PY{n}{producerMessage}\PY{p}{.}\PY{n}{unitNumber} \PY{o}{=} \PY{n}{get\PYZus{}random}\PY{p}{(}\PY{p}{)}\PY{p}{;}
        \PY{n}{producerMessage}\PY{p}{.}\PY{n}{workTime} \PY{o}{=} \PY{n}{get\PYZus{}random\PYZus{}between}\PY{p}{(}\PY{l+m+mi}{2}\PY{p}{,} \PY{l+m+mi}{9}\PY{p}{)}\PY{p}{;}
        \PY{n}{producerWaitTime} \PY{o}{=} \PY{n}{get\PYZus{}random\PYZus{}between}\PY{p}{(}\PY{l+m+mi}{3}\PY{p}{,} \PY{l+m+mi}{7}\PY{p}{)}\PY{p}{;}
        \PY{n}{write}\PY{p}{(}\PY{n}{writePipe}\PY{p}{,} \PY{o}{\PYZam{}}\PY{n}{producerMessage}\PY{p}{,} \PY{k}{sizeof}\PY{p}{(}\PY{k}{struct} \PY{n}{WorkUnit}\PY{p}{)}\PY{p}{)}\PY{p}{;}
        \PY{n}{sleep}\PY{p}{(}\PY{n}{producerWaitTime}\PY{p}{)}\PY{p}{;}
    \PY{p}{\PYZcb{}}
\PY{p}{\PYZcb{}}

\PY{c+cp}{\PYZsh{}}\PY{c+cp}{pragma clang diagnostic pop}

\PY{k+kt}{int} \PY{n+nf}{main}\PY{p}{(}\PY{k+kt}{int} \PY{n}{argc}\PY{p}{,} \PY{k+kt}{char} \PY{o}{*}\PY{n}{argv}\PY{p}{[}\PY{p}{]}\PY{p}{)} \PY{p}{\PYZob{}}
    \PY{k+kt}{pthread\PYZus{}t} \PY{n}{threads}\PY{p}{[}\PY{n}{NUM\PYZus{}THREADS}\PY{p}{]}\PY{p}{;}
    \PY{k+kt}{int} \PY{n}{rc}\PY{p}{;}
    \PY{k+kt}{long} \PY{n}{t}\PY{p}{;}

    \PY{n}{pipe}\PY{p}{(}\PY{n}{workPipe}\PY{p}{)}\PY{p}{;}

    \PY{k}{for} \PY{p}{(}\PY{n}{t} \PY{o}{=} \PY{l+m+mi}{0}\PY{p}{;} \PY{n}{t} \PY{o}{\PYZlt{}} \PY{n}{NUM\PYZus{}THREADS}\PY{p}{;} \PY{n}{t}\PY{o}{+}\PY{o}{+}\PY{p}{)} \PY{p}{\PYZob{}}
        \PY{n}{printf}\PY{p}{(}\PY{l+s}{\PYZdq{}}\PY{l+s}{In main: creating thread \PYZpc{}ld}\PY{l+s+se}{\PYZbs{}n}\PY{l+s}{\PYZdq{}}\PY{p}{,} \PY{n}{t}\PY{p}{)}\PY{p}{;}
        \PY{n}{rc} \PY{o}{=} \PY{n}{pthread\PYZus{}create}\PY{p}{(}\PY{o}{\PYZam{}}\PY{n}{threads}\PY{p}{[}\PY{n}{t}\PY{p}{]}\PY{p}{,} \PY{n+nb}{NULL}\PY{p}{,} \PY{n}{consumer}\PY{p}{,} \PY{p}{(}\PY{k+kt}{void} \PY{o}{*}\PY{p}{)} \PY{n}{t}\PY{p}{)}\PY{p}{;}
        \PY{k}{if} \PY{p}{(}\PY{n}{rc}\PY{p}{)} \PY{p}{\PYZob{}}
            \PY{n}{printf}\PY{p}{(}\PY{l+s}{\PYZdq{}}\PY{l+s}{ERROR; return code from pthread\PYZus{}create() is \PYZpc{}d}\PY{l+s+se}{\PYZbs{}n}\PY{l+s}{\PYZdq{}}\PY{p}{,} \PY{n}{rc}\PY{p}{)}\PY{p}{;}
            \PY{n}{exit}\PY{p}{(}\PY{o}{\PYZhy{}}\PY{l+m+mi}{1}\PY{p}{)}\PY{p}{;}
        \PY{p}{\PYZcb{}}
    \PY{p}{\PYZcb{}}

    \PY{n}{producer}\PY{p}{(}\PY{p}{)}\PY{p}{;}
    \PY{n}{pthread\PYZus{}exit}\PY{p}{(}\PY{n+nb}{NULL}\PY{p}{)}\PY{p}{;}
\PY{p}{\PYZcb{}}
\end{Verbatim}


\subsection{get\_random.c}

\begin{Verbatim}[commandchars=\\\{\}]
\PY{c+cp}{\PYZsh{}}\PY{c+cp}{include \PYZlt{}stdio.h\PYZgt{}}
\PY{c+cp}{\PYZsh{}}\PY{c+cp}{include \PYZdq{}using\PYZus{}drng.h\PYZdq{}}

\PY{c+cp}{\PYZsh{}}\PY{c+cp}{if USING\PYZus{}DRNG}
\PY{c+cp}{\PYZsh{}}\PY{c+cp}{include \PYZdq{}vendor}\PY{c+cp}{/}\PY{c+cp}{drng}\PY{c+cp}{/}\PY{c+cp}{drng.h\PYZdq{}}
\PY{c+cp}{\PYZsh{}}\PY{c+cp}{else}

\PY{c+cp}{\PYZsh{}}\PY{c+cp}{include \PYZdq{}vendor}\PY{c+cp}{/}\PY{c+cp}{mt19937ar.h\PYZdq{}}

\PY{c+cp}{\PYZsh{}}\PY{c+cp}{endif}

\PY{k+kt}{unsigned} \PY{k+kt}{int} \PY{n+nf}{get\PYZus{}random}\PY{p}{(}\PY{p}{)} \PY{p}{\PYZob{}}
\PY{c+cp}{\PYZsh{}}\PY{c+cp}{if USING\PYZus{}DRNG}
    \PY{c+c1}{// Use intel drandr}
    \PY{k+kt}{uint32\PYZus{}t} \PY{n}{rand}\PY{p}{[}\PY{l+m+mi}{1}\PY{p}{]}\PY{p}{;}
    \PY{k}{if} \PY{p}{(}\PY{n}{rdrand\PYZus{}get\PYZus{}n\PYZus{}uints}\PY{p}{(}\PY{l+m+mi}{1}\PY{p}{,} \PY{n}{rand}\PY{p}{)} \PY{o}{=}\PY{o}{=} \PY{l+m+mi}{0}\PY{p}{)} \PY{p}{\PYZob{}}
        \PY{n}{fprintf}\PY{p}{(}\PY{n}{stderr}\PY{p}{,} \PY{l+s}{\PYZdq{}}\PY{l+s}{Random values not available}\PY{l+s+se}{\PYZbs{}n}\PY{l+s}{\PYZdq{}}\PY{p}{)}\PY{p}{;}
        \PY{k}{return} \PY{l+m+mi}{1}\PY{p}{;}
    \PY{p}{\PYZcb{}}
    \PY{k}{return} \PY{p}{(}\PY{k+kt}{unsigned} \PY{k+kt}{int}\PY{p}{)} \PY{n}{rand}\PY{p}{[}\PY{l+m+mi}{0}\PY{p}{]}\PY{p}{;}
\PY{c+cp}{\PYZsh{}}\PY{c+cp}{else}
    \PY{c+c1}{// Use twister}
    \PY{k}{return} \PY{p}{(}\PY{k+kt}{unsigned} \PY{k+kt}{int}\PY{p}{)} \PY{n}{genrand\PYZus{}int32}\PY{p}{(}\PY{p}{)}\PY{p}{;}
\PY{c+cp}{\PYZsh{}}\PY{c+cp}{endif}
\PY{p}{\PYZcb{}}

\PY{k+kt}{unsigned} \PY{k+kt}{int} \PY{n+nf}{get\PYZus{}random\PYZus{}between}\PY{p}{(}\PY{k+kt}{int} \PY{n}{min}\PY{p}{,} \PY{k+kt}{int} \PY{n}{max}\PY{p}{)} \PY{p}{\PYZob{}}
    \PY{k}{return} \PY{n}{get\PYZus{}random}\PY{p}{(}\PY{p}{)} \PY{o}{\PYZpc{}} \PY{p}{(}\PY{n}{max} \PY{o}{\PYZhy{}} \PY{n}{min} \PY{o}{+} \PY{l+m+mi}{1}\PY{p}{)} \PY{o}{+} \PY{n}{min}\PY{p}{;}
\PY{p}{\PYZcb{}}
\end{Verbatim}


\subsection{mt19937ar.c}

\begin{Verbatim}[commandchars=\\\{\}]
\PY{c+cm}{/* }
\PY{c+cm}{   A C\PYZhy{}program for MT19937, with initialization improved 2002/1/26.}
\PY{c+cm}{   Coded by Takuji Nishimura and Makoto Matsumoto.}

\PY{c+cm}{   Before using, initialize the state by using init\PYZus{}genrand(seed)  }
\PY{c+cm}{   or init\PYZus{}by\PYZus{}array(init\PYZus{}key, key\PYZus{}length).}

\PY{c+cm}{   Copyright (C) 1997 \PYZhy{} 2002, Makoto Matsumoto and Takuji Nishimura,}
\PY{c+cm}{   All rights reserved.                          }
\PY{c+cm}{   Copyright (C) 2005, Mutsuo Saito,}
\PY{c+cm}{   All rights reserved.                          }

\PY{c+cm}{   Redistribution and use in source and binary forms, with or without}
\PY{c+cm}{   modification, are permitted provided that the following conditions}
\PY{c+cm}{   are met:}

\PY{c+cm}{     1. Redistributions of source code must retain the above copyright}
\PY{c+cm}{        notice, this list of conditions and the following disclaimer.}

\PY{c+cm}{     2. Redistributions in binary form must reproduce the above copyright}
\PY{c+cm}{        notice, this list of conditions and the following disclaimer in the}
\PY{c+cm}{        documentation and/or other materials provided with the distribution.}

\PY{c+cm}{     3. The names of its contributors may not be used to endorse or promote }
\PY{c+cm}{        products derived from this software without specific prior written }
\PY{c+cm}{        permission.}

\PY{c+cm}{   THIS SOFTWARE IS PROVIDED BY THE COPYRIGHT HOLDERS AND CONTRIBUTORS}
\PY{c+cm}{   \PYZdq{}AS IS\PYZdq{} AND ANY EXPRESS OR IMPLIED WARRANTIES, INCLUDING, BUT NOT}
\PY{c+cm}{   LIMITED TO, THE IMPLIED WARRANTIES OF MERCHANTABILITY AND FITNESS FOR}
\PY{c+cm}{   A PARTICULAR PURPOSE ARE DISCLAIMED.  IN NO EVENT SHALL THE COPYRIGHT OWNER OR}
\PY{c+cm}{   CONTRIBUTORS BE LIABLE FOR ANY DIRECT, INDIRECT, INCIDENTAL, SPECIAL,}
\PY{c+cm}{   EXEMPLARY, OR CONSEQUENTIAL DAMAGES (INCLUDING, BUT NOT LIMITED TO,}
\PY{c+cm}{   PROCUREMENT OF SUBSTITUTE GOODS OR SERVICES; LOSS OF USE, DATA, OR}
\PY{c+cm}{   PROFITS; OR BUSINESS INTERRUPTION) HOWEVER CAUSED AND ON ANY THEORY OF}
\PY{c+cm}{   LIABILITY, WHETHER IN CONTRACT, STRICT LIABILITY, OR TORT (INCLUDING}
\PY{c+cm}{   NEGLIGENCE OR OTHERWISE) ARISING IN ANY WAY OUT OF THE USE OF THIS}
\PY{c+cm}{   SOFTWARE, EVEN IF ADVISED OF THE POSSIBILITY OF SUCH DAMAGE.}


\PY{c+cm}{   Any feedback is very welcome.}
\PY{c+cm}{   http://www.math.sci.hiroshima\PYZhy{}u.ac.jp/\PYZti{}m\PYZhy{}mat/MT/emt.html}
\PY{c+cm}{   email: m\PYZhy{}mat @ math.sci.hiroshima\PYZhy{}u.ac.jp (remove space)}
\PY{c+cm}{*/}

\PY{c+cp}{\PYZsh{}}\PY{c+cp}{include \PYZlt{}stdio.h\PYZgt{}}
\PY{c+cp}{\PYZsh{}}\PY{c+cp}{include \PYZdq{}mt19937ar.h\PYZdq{}}

\PY{c+cm}{/* Period parameters */}  
\PY{c+cp}{\PYZsh{}}\PY{c+cp}{define N 624}
\PY{c+cp}{\PYZsh{}}\PY{c+cp}{define M 397}
\PY{c+cp}{\PYZsh{}}\PY{c+cp}{define MATRIX\PYZus{}A 0x9908b0dfUL   }\PY{c+cm}{/* constant vector a */}
\PY{c+cp}{\PYZsh{}}\PY{c+cp}{define UPPER\PYZus{}MASK 0x80000000UL }\PY{c+cm}{/* most significant w\PYZhy{}r bits */}
\PY{c+cp}{\PYZsh{}}\PY{c+cp}{define LOWER\PYZus{}MASK 0x7fffffffUL }\PY{c+cm}{/* least significant r bits */}

\PY{k}{static} \PY{k+kt}{unsigned} \PY{k+kt}{long} \PY{n}{mt}\PY{p}{[}\PY{n}{N}\PY{p}{]}\PY{p}{;} \PY{c+cm}{/* the array for the state vector  */}
\PY{k}{static} \PY{k+kt}{int} \PY{n}{mti}\PY{o}{=}\PY{n}{N}\PY{o}{+}\PY{l+m+mi}{1}\PY{p}{;} \PY{c+cm}{/* mti==N+1 means mt[N] is not initialized */}

\PY{c+cm}{/* initializes mt[N] with a seed */}
\PY{k+kt}{void} \PY{n+nf}{init\PYZus{}genrand}\PY{p}{(}\PY{k+kt}{unsigned} \PY{k+kt}{long} \PY{n}{s}\PY{p}{)}
\PY{p}{\PYZob{}}
    \PY{n}{mt}\PY{p}{[}\PY{l+m+mi}{0}\PY{p}{]}\PY{o}{=} \PY{n}{s} \PY{o}{\PYZam{}} \PY{l+m+mh}{0xffffffffUL}\PY{p}{;}
    \PY{k}{for} \PY{p}{(}\PY{n}{mti}\PY{o}{=}\PY{l+m+mi}{1}\PY{p}{;} \PY{n}{mti}\PY{o}{\PYZlt{}}\PY{n}{N}\PY{p}{;} \PY{n}{mti}\PY{o}{+}\PY{o}{+}\PY{p}{)} \PY{p}{\PYZob{}}
        \PY{n}{mt}\PY{p}{[}\PY{n}{mti}\PY{p}{]} \PY{o}{=} 
	    \PY{p}{(}\PY{l+m+mi}{1812433253UL} \PY{o}{*} \PY{p}{(}\PY{n}{mt}\PY{p}{[}\PY{n}{mti}\PY{o}{\PYZhy{}}\PY{l+m+mi}{1}\PY{p}{]} \PY{o}{\PYZca{}} \PY{p}{(}\PY{n}{mt}\PY{p}{[}\PY{n}{mti}\PY{o}{\PYZhy{}}\PY{l+m+mi}{1}\PY{p}{]} \PY{o}{\PYZgt{}}\PY{o}{\PYZgt{}} \PY{l+m+mi}{30}\PY{p}{)}\PY{p}{)} \PY{o}{+} \PY{n}{mti}\PY{p}{)}\PY{p}{;} 
        \PY{c+cm}{/* See Knuth TAOCP Vol2. 3rd Ed. P.106 for multiplier. */}
        \PY{c+cm}{/* In the previous versions, MSBs of the seed affect   */}
        \PY{c+cm}{/* only MSBs of the array mt[].                        */}
        \PY{c+cm}{/* 2002/01/09 modified by Makoto Matsumoto             */}
        \PY{n}{mt}\PY{p}{[}\PY{n}{mti}\PY{p}{]} \PY{o}{\PYZam{}}\PY{o}{=} \PY{l+m+mh}{0xffffffffUL}\PY{p}{;}
        \PY{c+cm}{/* for \PYZgt{}32 bit machines */}
    \PY{p}{\PYZcb{}}
\PY{p}{\PYZcb{}}

\PY{c+cm}{/* initialize by an array with array\PYZhy{}length */}
\PY{c+cm}{/* init\PYZus{}key is the array for initializing keys */}
\PY{c+cm}{/* key\PYZus{}length is its length */}
\PY{c+cm}{/* slight change for C++, 2004/2/26 */}
\PY{k+kt}{void} \PY{n+nf}{init\PYZus{}by\PYZus{}array}\PY{p}{(}\PY{k+kt}{unsigned} \PY{k+kt}{long} \PY{n}{init\PYZus{}key}\PY{p}{[}\PY{p}{]}\PY{p}{,} \PY{k+kt}{int} \PY{n}{key\PYZus{}length}\PY{p}{)}
\PY{p}{\PYZob{}}
    \PY{k+kt}{int} \PY{n}{i}\PY{p}{,} \PY{n}{j}\PY{p}{,} \PY{n}{k}\PY{p}{;}
    \PY{n}{init\PYZus{}genrand}\PY{p}{(}\PY{l+m+mi}{19650218UL}\PY{p}{)}\PY{p}{;}
    \PY{n}{i}\PY{o}{=}\PY{l+m+mi}{1}\PY{p}{;} \PY{n}{j}\PY{o}{=}\PY{l+m+mi}{0}\PY{p}{;}
    \PY{n}{k} \PY{o}{=} \PY{p}{(}\PY{n}{N}\PY{o}{\PYZgt{}}\PY{n}{key\PYZus{}length} \PY{o}{?} \PY{n+nl}{N} \PY{p}{:} \PY{n}{key\PYZus{}length}\PY{p}{)}\PY{p}{;}
    \PY{k}{for} \PY{p}{(}\PY{p}{;} \PY{n}{k}\PY{p}{;} \PY{n}{k}\PY{o}{\PYZhy{}}\PY{o}{\PYZhy{}}\PY{p}{)} \PY{p}{\PYZob{}}
        \PY{n}{mt}\PY{p}{[}\PY{n}{i}\PY{p}{]} \PY{o}{=} \PY{p}{(}\PY{n}{mt}\PY{p}{[}\PY{n}{i}\PY{p}{]} \PY{o}{\PYZca{}} \PY{p}{(}\PY{p}{(}\PY{n}{mt}\PY{p}{[}\PY{n}{i}\PY{o}{\PYZhy{}}\PY{l+m+mi}{1}\PY{p}{]} \PY{o}{\PYZca{}} \PY{p}{(}\PY{n}{mt}\PY{p}{[}\PY{n}{i}\PY{o}{\PYZhy{}}\PY{l+m+mi}{1}\PY{p}{]} \PY{o}{\PYZgt{}}\PY{o}{\PYZgt{}} \PY{l+m+mi}{30}\PY{p}{)}\PY{p}{)} \PY{o}{*} \PY{l+m+mi}{1664525UL}\PY{p}{)}\PY{p}{)}
          \PY{o}{+} \PY{n}{init\PYZus{}key}\PY{p}{[}\PY{n}{j}\PY{p}{]} \PY{o}{+} \PY{n}{j}\PY{p}{;} \PY{c+cm}{/* non linear */}
        \PY{n}{mt}\PY{p}{[}\PY{n}{i}\PY{p}{]} \PY{o}{\PYZam{}}\PY{o}{=} \PY{l+m+mh}{0xffffffffUL}\PY{p}{;} \PY{c+cm}{/* for WORDSIZE \PYZgt{} 32 machines */}
        \PY{n}{i}\PY{o}{+}\PY{o}{+}\PY{p}{;} \PY{n}{j}\PY{o}{+}\PY{o}{+}\PY{p}{;}
        \PY{k}{if} \PY{p}{(}\PY{n}{i}\PY{o}{\PYZgt{}}\PY{o}{=}\PY{n}{N}\PY{p}{)} \PY{p}{\PYZob{}} \PY{n}{mt}\PY{p}{[}\PY{l+m+mi}{0}\PY{p}{]} \PY{o}{=} \PY{n}{mt}\PY{p}{[}\PY{n}{N}\PY{o}{\PYZhy{}}\PY{l+m+mi}{1}\PY{p}{]}\PY{p}{;} \PY{n}{i}\PY{o}{=}\PY{l+m+mi}{1}\PY{p}{;} \PY{p}{\PYZcb{}}
        \PY{k}{if} \PY{p}{(}\PY{n}{j}\PY{o}{\PYZgt{}}\PY{o}{=}\PY{n}{key\PYZus{}length}\PY{p}{)} \PY{n}{j}\PY{o}{=}\PY{l+m+mi}{0}\PY{p}{;}
    \PY{p}{\PYZcb{}}
    \PY{k}{for} \PY{p}{(}\PY{n}{k}\PY{o}{=}\PY{n}{N}\PY{o}{\PYZhy{}}\PY{l+m+mi}{1}\PY{p}{;} \PY{n}{k}\PY{p}{;} \PY{n}{k}\PY{o}{\PYZhy{}}\PY{o}{\PYZhy{}}\PY{p}{)} \PY{p}{\PYZob{}}
        \PY{n}{mt}\PY{p}{[}\PY{n}{i}\PY{p}{]} \PY{o}{=} \PY{p}{(}\PY{n}{mt}\PY{p}{[}\PY{n}{i}\PY{p}{]} \PY{o}{\PYZca{}} \PY{p}{(}\PY{p}{(}\PY{n}{mt}\PY{p}{[}\PY{n}{i}\PY{o}{\PYZhy{}}\PY{l+m+mi}{1}\PY{p}{]} \PY{o}{\PYZca{}} \PY{p}{(}\PY{n}{mt}\PY{p}{[}\PY{n}{i}\PY{o}{\PYZhy{}}\PY{l+m+mi}{1}\PY{p}{]} \PY{o}{\PYZgt{}}\PY{o}{\PYZgt{}} \PY{l+m+mi}{30}\PY{p}{)}\PY{p}{)} \PY{o}{*} \PY{l+m+mi}{1566083941UL}\PY{p}{)}\PY{p}{)}
          \PY{o}{\PYZhy{}} \PY{n}{i}\PY{p}{;} \PY{c+cm}{/* non linear */}
        \PY{n}{mt}\PY{p}{[}\PY{n}{i}\PY{p}{]} \PY{o}{\PYZam{}}\PY{o}{=} \PY{l+m+mh}{0xffffffffUL}\PY{p}{;} \PY{c+cm}{/* for WORDSIZE \PYZgt{} 32 machines */}
        \PY{n}{i}\PY{o}{+}\PY{o}{+}\PY{p}{;}
        \PY{k}{if} \PY{p}{(}\PY{n}{i}\PY{o}{\PYZgt{}}\PY{o}{=}\PY{n}{N}\PY{p}{)} \PY{p}{\PYZob{}} \PY{n}{mt}\PY{p}{[}\PY{l+m+mi}{0}\PY{p}{]} \PY{o}{=} \PY{n}{mt}\PY{p}{[}\PY{n}{N}\PY{o}{\PYZhy{}}\PY{l+m+mi}{1}\PY{p}{]}\PY{p}{;} \PY{n}{i}\PY{o}{=}\PY{l+m+mi}{1}\PY{p}{;} \PY{p}{\PYZcb{}}
    \PY{p}{\PYZcb{}}

    \PY{n}{mt}\PY{p}{[}\PY{l+m+mi}{0}\PY{p}{]} \PY{o}{=} \PY{l+m+mh}{0x80000000UL}\PY{p}{;} \PY{c+cm}{/* MSB is 1; assuring non\PYZhy{}zero initial array */} 
\PY{p}{\PYZcb{}}

\PY{c+cm}{/* generates a random number on [0,0xffffffff]\PYZhy{}interval */}
\PY{k+kt}{unsigned} \PY{k+kt}{long} \PY{n+nf}{genrand\PYZus{}int32}\PY{p}{(}\PY{k+kt}{void}\PY{p}{)}
\PY{p}{\PYZob{}}
    \PY{k+kt}{unsigned} \PY{k+kt}{long} \PY{n}{y}\PY{p}{;}
    \PY{k}{static} \PY{k+kt}{unsigned} \PY{k+kt}{long} \PY{n}{mag01}\PY{p}{[}\PY{l+m+mi}{2}\PY{p}{]}\PY{o}{=}\PY{p}{\PYZob{}}\PY{l+m+mh}{0x0UL}\PY{p}{,} \PY{n}{MATRIX\PYZus{}A}\PY{p}{\PYZcb{}}\PY{p}{;}
    \PY{c+cm}{/* mag01[x] = x * MATRIX\PYZus{}A  for x=0,1 */}

    \PY{k}{if} \PY{p}{(}\PY{n}{mti} \PY{o}{\PYZgt{}}\PY{o}{=} \PY{n}{N}\PY{p}{)} \PY{p}{\PYZob{}} \PY{c+cm}{/* generate N words at one time */}
        \PY{k+kt}{int} \PY{n}{kk}\PY{p}{;}

        \PY{k}{if} \PY{p}{(}\PY{n}{mti} \PY{o}{=}\PY{o}{=} \PY{n}{N}\PY{o}{+}\PY{l+m+mi}{1}\PY{p}{)}   \PY{c+cm}{/* if init\PYZus{}genrand() has not been called, */}
            \PY{n}{init\PYZus{}genrand}\PY{p}{(}\PY{l+m+mi}{5489UL}\PY{p}{)}\PY{p}{;} \PY{c+cm}{/* a default initial seed is used */}

        \PY{k}{for} \PY{p}{(}\PY{n}{kk}\PY{o}{=}\PY{l+m+mi}{0}\PY{p}{;}\PY{n}{kk}\PY{o}{\PYZlt{}}\PY{n}{N}\PY{o}{\PYZhy{}}\PY{n}{M}\PY{p}{;}\PY{n}{kk}\PY{o}{+}\PY{o}{+}\PY{p}{)} \PY{p}{\PYZob{}}
            \PY{n}{y} \PY{o}{=} \PY{p}{(}\PY{n}{mt}\PY{p}{[}\PY{n}{kk}\PY{p}{]}\PY{o}{\PYZam{}}\PY{n}{UPPER\PYZus{}MASK}\PY{p}{)}\PY{o}{|}\PY{p}{(}\PY{n}{mt}\PY{p}{[}\PY{n}{kk}\PY{o}{+}\PY{l+m+mi}{1}\PY{p}{]}\PY{o}{\PYZam{}}\PY{n}{LOWER\PYZus{}MASK}\PY{p}{)}\PY{p}{;}
            \PY{n}{mt}\PY{p}{[}\PY{n}{kk}\PY{p}{]} \PY{o}{=} \PY{n}{mt}\PY{p}{[}\PY{n}{kk}\PY{o}{+}\PY{n}{M}\PY{p}{]} \PY{o}{\PYZca{}} \PY{p}{(}\PY{n}{y} \PY{o}{\PYZgt{}}\PY{o}{\PYZgt{}} \PY{l+m+mi}{1}\PY{p}{)} \PY{o}{\PYZca{}} \PY{n}{mag01}\PY{p}{[}\PY{n}{y} \PY{o}{\PYZam{}} \PY{l+m+mh}{0x1UL}\PY{p}{]}\PY{p}{;}
        \PY{p}{\PYZcb{}}
        \PY{k}{for} \PY{p}{(}\PY{p}{;}\PY{n}{kk}\PY{o}{\PYZlt{}}\PY{n}{N}\PY{o}{\PYZhy{}}\PY{l+m+mi}{1}\PY{p}{;}\PY{n}{kk}\PY{o}{+}\PY{o}{+}\PY{p}{)} \PY{p}{\PYZob{}}
            \PY{n}{y} \PY{o}{=} \PY{p}{(}\PY{n}{mt}\PY{p}{[}\PY{n}{kk}\PY{p}{]}\PY{o}{\PYZam{}}\PY{n}{UPPER\PYZus{}MASK}\PY{p}{)}\PY{o}{|}\PY{p}{(}\PY{n}{mt}\PY{p}{[}\PY{n}{kk}\PY{o}{+}\PY{l+m+mi}{1}\PY{p}{]}\PY{o}{\PYZam{}}\PY{n}{LOWER\PYZus{}MASK}\PY{p}{)}\PY{p}{;}
            \PY{n}{mt}\PY{p}{[}\PY{n}{kk}\PY{p}{]} \PY{o}{=} \PY{n}{mt}\PY{p}{[}\PY{n}{kk}\PY{o}{+}\PY{p}{(}\PY{n}{M}\PY{o}{\PYZhy{}}\PY{n}{N}\PY{p}{)}\PY{p}{]} \PY{o}{\PYZca{}} \PY{p}{(}\PY{n}{y} \PY{o}{\PYZgt{}}\PY{o}{\PYZgt{}} \PY{l+m+mi}{1}\PY{p}{)} \PY{o}{\PYZca{}} \PY{n}{mag01}\PY{p}{[}\PY{n}{y} \PY{o}{\PYZam{}} \PY{l+m+mh}{0x1UL}\PY{p}{]}\PY{p}{;}
        \PY{p}{\PYZcb{}}
        \PY{n}{y} \PY{o}{=} \PY{p}{(}\PY{n}{mt}\PY{p}{[}\PY{n}{N}\PY{o}{\PYZhy{}}\PY{l+m+mi}{1}\PY{p}{]}\PY{o}{\PYZam{}}\PY{n}{UPPER\PYZus{}MASK}\PY{p}{)}\PY{o}{|}\PY{p}{(}\PY{n}{mt}\PY{p}{[}\PY{l+m+mi}{0}\PY{p}{]}\PY{o}{\PYZam{}}\PY{n}{LOWER\PYZus{}MASK}\PY{p}{)}\PY{p}{;}
        \PY{n}{mt}\PY{p}{[}\PY{n}{N}\PY{o}{\PYZhy{}}\PY{l+m+mi}{1}\PY{p}{]} \PY{o}{=} \PY{n}{mt}\PY{p}{[}\PY{n}{M}\PY{o}{\PYZhy{}}\PY{l+m+mi}{1}\PY{p}{]} \PY{o}{\PYZca{}} \PY{p}{(}\PY{n}{y} \PY{o}{\PYZgt{}}\PY{o}{\PYZgt{}} \PY{l+m+mi}{1}\PY{p}{)} \PY{o}{\PYZca{}} \PY{n}{mag01}\PY{p}{[}\PY{n}{y} \PY{o}{\PYZam{}} \PY{l+m+mh}{0x1UL}\PY{p}{]}\PY{p}{;}

        \PY{n}{mti} \PY{o}{=} \PY{l+m+mi}{0}\PY{p}{;}
    \PY{p}{\PYZcb{}}
  
    \PY{n}{y} \PY{o}{=} \PY{n}{mt}\PY{p}{[}\PY{n}{mti}\PY{o}{+}\PY{o}{+}\PY{p}{]}\PY{p}{;}

    \PY{c+cm}{/* Tempering */}
    \PY{n}{y} \PY{o}{\PYZca{}}\PY{o}{=} \PY{p}{(}\PY{n}{y} \PY{o}{\PYZgt{}}\PY{o}{\PYZgt{}} \PY{l+m+mi}{11}\PY{p}{)}\PY{p}{;}
    \PY{n}{y} \PY{o}{\PYZca{}}\PY{o}{=} \PY{p}{(}\PY{n}{y} \PY{o}{\PYZlt{}}\PY{o}{\PYZlt{}} \PY{l+m+mi}{7}\PY{p}{)} \PY{o}{\PYZam{}} \PY{l+m+mh}{0x9d2c5680UL}\PY{p}{;}
    \PY{n}{y} \PY{o}{\PYZca{}}\PY{o}{=} \PY{p}{(}\PY{n}{y} \PY{o}{\PYZlt{}}\PY{o}{\PYZlt{}} \PY{l+m+mi}{15}\PY{p}{)} \PY{o}{\PYZam{}} \PY{l+m+mh}{0xefc60000UL}\PY{p}{;}
    \PY{n}{y} \PY{o}{\PYZca{}}\PY{o}{=} \PY{p}{(}\PY{n}{y} \PY{o}{\PYZgt{}}\PY{o}{\PYZgt{}} \PY{l+m+mi}{18}\PY{p}{)}\PY{p}{;}

    \PY{k}{return} \PY{n}{y}\PY{p}{;}
\PY{p}{\PYZcb{}}

\PY{c+cm}{/* generates a random number on [0,0x7fffffff]\PYZhy{}interval */}
\PY{k+kt}{long} \PY{n+nf}{genrand\PYZus{}int31}\PY{p}{(}\PY{k+kt}{void}\PY{p}{)}
\PY{p}{\PYZob{}}
    \PY{k}{return} \PY{p}{(}\PY{k+kt}{long}\PY{p}{)}\PY{p}{(}\PY{n}{genrand\PYZus{}int32}\PY{p}{(}\PY{p}{)}\PY{o}{\PYZgt{}}\PY{o}{\PYZgt{}}\PY{l+m+mi}{1}\PY{p}{)}\PY{p}{;}
\PY{p}{\PYZcb{}}

\PY{c+cm}{/* generates a random number on [0,1]\PYZhy{}real\PYZhy{}interval */}
\PY{k+kt}{double} \PY{n+nf}{genrand\PYZus{}real1}\PY{p}{(}\PY{k+kt}{void}\PY{p}{)}
\PY{p}{\PYZob{}}
    \PY{k}{return} \PY{n}{genrand\PYZus{}int32}\PY{p}{(}\PY{p}{)}\PY{o}{*}\PY{p}{(}\PY{l+m+mf}{1.0}\PY{o}{/}\PY{l+m+mf}{4294967295.0}\PY{p}{)}\PY{p}{;} 
    \PY{c+cm}{/* divided by 2\PYZca{}32\PYZhy{}1 */} 
\PY{p}{\PYZcb{}}

\PY{c+cm}{/* generates a random number on [0,1)\PYZhy{}real\PYZhy{}interval */}
\PY{k+kt}{double} \PY{n+nf}{genrand\PYZus{}real2}\PY{p}{(}\PY{k+kt}{void}\PY{p}{)}
\PY{p}{\PYZob{}}
    \PY{k}{return} \PY{n}{genrand\PYZus{}int32}\PY{p}{(}\PY{p}{)}\PY{o}{*}\PY{p}{(}\PY{l+m+mf}{1.0}\PY{o}{/}\PY{l+m+mf}{4294967296.0}\PY{p}{)}\PY{p}{;} 
    \PY{c+cm}{/* divided by 2\PYZca{}32 */}
\PY{p}{\PYZcb{}}

\PY{c+cm}{/* generates a random number on (0,1)\PYZhy{}real\PYZhy{}interval */}
\PY{k+kt}{double} \PY{n+nf}{genrand\PYZus{}real3}\PY{p}{(}\PY{k+kt}{void}\PY{p}{)}
\PY{p}{\PYZob{}}
    \PY{k}{return} \PY{p}{(}\PY{p}{(}\PY{p}{(}\PY{k+kt}{double}\PY{p}{)}\PY{n}{genrand\PYZus{}int32}\PY{p}{(}\PY{p}{)}\PY{p}{)} \PY{o}{+} \PY{l+m+mf}{0.5}\PY{p}{)}\PY{o}{*}\PY{p}{(}\PY{l+m+mf}{1.0}\PY{o}{/}\PY{l+m+mf}{4294967296.0}\PY{p}{)}\PY{p}{;} 
    \PY{c+cm}{/* divided by 2\PYZca{}32 */}
\PY{p}{\PYZcb{}}

\PY{c+cm}{/* generates a random number on [0,1) with 53\PYZhy{}bit resolution*/}
\PY{k+kt}{double} \PY{n+nf}{genrand\PYZus{}res53}\PY{p}{(}\PY{k+kt}{void}\PY{p}{)} 
\PY{p}{\PYZob{}} 
    \PY{k+kt}{unsigned} \PY{k+kt}{long} \PY{n}{a}\PY{o}{=}\PY{n}{genrand\PYZus{}int32}\PY{p}{(}\PY{p}{)}\PY{o}{\PYZgt{}}\PY{o}{\PYZgt{}}\PY{l+m+mi}{5}\PY{p}{,} \PY{n}{b}\PY{o}{=}\PY{n}{genrand\PYZus{}int32}\PY{p}{(}\PY{p}{)}\PY{o}{\PYZgt{}}\PY{o}{\PYZgt{}}\PY{l+m+mi}{6}\PY{p}{;} 
    \PY{k}{return}\PY{p}{(}\PY{n}{a}\PY{o}{*}\PY{l+m+mf}{67108864.0}\PY{o}{+}\PY{n}{b}\PY{p}{)}\PY{o}{*}\PY{p}{(}\PY{l+m+mf}{1.0}\PY{o}{/}\PY{l+m+mf}{9007199254740992.0}\PY{p}{)}\PY{p}{;} 
\PY{p}{\PYZcb{}} 
\PY{c+cm}{/* These real versions are due to Isaku Wada, 2002/01/09 added */}
\end{Verbatim}


\end{document}
